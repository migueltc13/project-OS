\documentclass[a4paper,11pt]{scrreprt}
    %% Used for changing geometry of the page
    %% Cover page text cannot overlay cover sketching/style
    %% https://ctan.org/pkg/geometry?lang=en
\usepackage{geometry}
    %% Changes language of some packages protocols
    %% e.g., when captioning images: Figure 1. -> Figura 1.
    %% https://ctan.org/pkg/babel?lang=en
\usepackage[portuguese]{babel}
    %% Used for special fonts
    %% Cannot be compiled with pdflatex
    %% https://ctan.org/pkg/fontspec?lang=en
\usepackage{fontspec}
    %% Arial FONT
    \setmainfont{Arial}
    %% More colors and color options
    %% https://ctan.org/pkg/xcolor?lang=en
    %% https://ctan.org/pkg/colortbl?lang=en
\usepackage{xcolor,colortbl}
    %% More tabular options, like dashed/dotted lines
    %% https://ctan.org/pkg/arydshln?lang=en
\usepackage{arydshln}
    %% List of acronyms
    %% https://ctan.org/pkg/nomencl?lang=en
\usepackage[intoc]{nomencl}
    %% Must be called to init nomencl environment
    \makenomenclature
    %% More images options/settings
    %% https://ctan.org/pkg/graphicx?lang=en
\usepackage{graphics}
    %% Defining subdirectories to image path enviornment
    %% \graphicspath{{sub1}{sub2}...{subN}}
    \graphicspath{{images}}

    %% used to handle cross-referencing commands in LaTeX to produce hypertext links in the document
    %% https://ctan.org/pkg/hyperref?lang=en
\usepackage{hyperref}
    %% math environments
    %% https://ctan.org/pkg/amsmath?lang=en
    %% settings
    \hypersetup{
        colorlinks,
        citecolor=black,
        filecolor=black,
        linkcolor=black,
        urlcolor=black
    }

\usepackage{amsmath}
    %% Defining backgrouns, used to make the cover
    %% https://ctan.org/pkg/background?lang=en
\usepackage[some]{background}
    %% Used to make drawings or complex graphics
    %% http://pgf.sourceforge.net/pgf_CVS.pdf
\usepackage{tikz}
    %% Tikz library to point operations ((x1,y1) + (x2,y2))
    \usetikzlibrary{calc}

%% code snippets
\usepackage{listings}
\usepackage{color}

\definecolor{dkgreen}{rgb}{0,0.6,0}
\definecolor{gray}{rgb}{0.5,0.5,0.5}
\definecolor{mauve}{rgb}{0.58,0,0.82}

\lstset{
    frame=tb,
    language=, % C, sh
    aboveskip=3mm,
    belowskip=3mm,
    showstringspaces=false,
    columns=flexible,
    basicstyle={\small\ttfamily},
    numbers=none,
    numberstyle=\small\color{gray},
    keywordstyle=\color{blue},
    commentstyle=\color{dkgreen},
    stringstyle=\color{mauve},
    breaklines=true,
    breakatwhitespace=true,
    tabsize=3,
    moredelim=**[is][\color{blue}]{@}{@}
}

%% further RelaX definitions
%% \usepackage{stix}

%% Defining sfdefault font and default font for document
\renewcommand{\familydefault}{\sfdefault}

%% For tables
\usepackage{pbox}
\usepackage{longtable}
\usepackage{xcolor} % for coloring rows
\usepackage{multirow}
\usepackage{hhline}
\usepackage{array}
\usepackage{diagbox}

% Custom commands for policy evaluation
\newcommand{\Exec}{
    \cellcolor{green!40}
}
\newcommand{\Wait}{
    \cellcolor{red!40}
}
\newcommand{\End}{
    \cellcolor{gray!40}
}

%==========================================================================
% DOCUMENT
%==========================================================================

\begin{document}

\pagenumbering{gobble}

%% Costume made cover
%% From there you can use \makecover command to build the cover
%% Blue cover color
\definecolor{titlepagecolor}{RGB}{60, 60, 60}

%==========================================================================
% COLORED BAR ON THE LEFT SIDE
%==========================================================================

\backgroundsetup{
    scale=1,
    angle=0,
    opacity=1,
    contents={
        \begin{tikzpicture}[remember picture,overlay]
            \path [fill=titlepagecolor] (-10.5,-15) rectangle ++ (5,30);
            \node[color=white] at (-6.9,-12) {\bfseries {\fontsize{120}{60} \textsf{S}}};
            \node[color=titlepagecolor] at (-3.9,-12) {\bfseries {\fontsize{120}{60} \textsf{O}}};
        \end{tikzpicture}
    }
}

%==========================================================================
% TITLE PAGE INFO
%==========================================================================

%% Changes values in this field to show information in the cover and back cover about your team/project

%% TITLE
\title{Orquestrador de Tarefas}

%% AUTHORS
\author{
    Flávia Alexandra Silva Araújo (A96587) \\
  \quad
    Miguel Torres Carvalho (A95485)
}

%% Date

\date{\today}

%% Course
\newcommand{\Course}{Licenciatura em Engenharia Informática}

%% Department
\newcommand{\Department}{Escola de Engenharia}

%% UniName
\newcommand{\UniName}{Universidade do Minho}

%% UniPic
\newcommand{\UniPic}{\includegraphics[scale=0.09]{images/uminho.png}}

%% University
\newcommand{\University}{
    \begin{flushleft}
        \UniPic
    \end{flushleft}
    \textcolor{gray}{\small\textbf{\textsf{\UniName}}}\par
    \textcolor{gray!80!white}{\small{\textsf{\Department}}}\par
    \textcolor{gray!70!white}{\small{\textsf{\Course}}}
}

%% UC
\newcommand{\UC}{
    \begin{flushleft}
        \par\textcolor{titlepagecolor}{  \LARGE\textbf{\textsf{Unidade Curricular de \\ Sistemas Operativos}}}
    \end{flushleft}
}

%% School Year
\newcommand{\SchoolYear}{
    \small{\textsf{Ano Letivo de 2023/2024}}}


%% Define new command to show title, author and date
\makeatletter
\let\Title\@title
\let\Author\@author
\let\Date\@date
\makeatother

%% MAKETEMPLATE
\newcommand{\makecover}{

%% Removes page number on footer
\thispagestyle{empty}

%% No indentation
\setlength{\parindent}{0em}

%% Put Background defined on \backgroundsetup, in this page
\BgThispage

%% Changing geometry to prevent overlay with text
%% At the end of back cover, geometry is default with \restoregeometry
\newgeometry{top=5cm,left=6cm,right=3cm,bottom=2cm}

%% builds university info defined previously
\University
\vspace{1cm}
%% builds curricular unity info defined previously
\UC
%% builds school year info defined previously
\SchoolYear

\vspace*{5cm}
%% bigger space (i think its the default one) between paragraphs
\setlength{\parskip}{1em}

%% builds title info defined previously
\par\textbf{\textsf{\huge\Title}}
\vspace{1cm}
%% builds author(s) info defined previously
\par\Author

\vspace{0.5cm}

%% builds date info defined previously
\par\Date
\restoregeometry
\pagebreak

}


% builds the cover
\makecover

%% smaller footer and header size
\newgeometry{top=2cm,left=3cm,right=3cm,bottom=3cm}
\savegeometry{default}

%==========================================================================
% BEGIN ABSTRACT PAGE
%==========================================================================

%% Abstract name: \Large font size, flushed left and paragraph skip before abstract content
\renewenvironment{abstract}
 {\par\noindent\textbf{\Large\abstractname}\par\bigskip}
 {}

\begin{flushleft}
\begin{abstract}
    No âmbito da unidade curricular de Sistemas Operativos,
    foi proposto o desenvolvimento de um serviço de execução assíncrona de tarefas,
    com a capacidade de encadear programas e consultar o estado das tarefas.

    Este serviço é composto por um servidor, denominado \textit{orchestrator},
    que recebe múltiplos pedidos de vários tipos dos clientes, e um \textit{client}, responsável
    por enviar pedidos de execução de tarefas, por consultar o estado das mesmas e por terminar o servidor.

    Ao longo deste trabalho, foram consolidados conceitos abordados na unidade curricular,
    como a utilização de \textit{FIFOs} para a comunicação entre processos, servidor e cliente;
    \textit{process forking} para a execução de comandos e consulta do estado das tarefas,
    sem que haja bloqueio do servidor;
    a utilização da função \textit{pipe} e redirecionamentos com a função \textit{dup2} para o encadeamento de comandos;
    o redirecionamento de descritores de ficheiros, de forma a registar o \textit{output} e os erros dos comandos,
    além da implementação de várias políticas de escalonamento de tarefas.

    Este documento descreve a um nível mais aprofundado a arquitetura do serviço desenvolvido,
    os módulos implementados, os argumentos da interface de linha de comandos, a avaliação das
    políticas de escalonamento implementadas e os testes realizados com respetivas reflexões.
\end{abstract}
\end{flushleft}

\pagebreak

%==========================================================================
% END ABSTRACT PAGE
%==========================================================================

%==========================================================================
% BEGIN INDEXES PAGES
%==========================================================================

%% Changes table of content name
%% Portuguese babel default : “Conteúdo”
%% Personally I prefer “índice”
\renewcommand{\contentsname}{Índice}
\renewcommand{\listfigurename}{Índice de Figuras}
\renewcommand{\listtablename}{Índice de Tabelas}

\begin{minipage}{\textwidth}
\tableofcontents
\listoffigures
\listoftables
\end{minipage}

%==========================================================================
% END INDEXES PAGES
%==========================================================================

%==========================================================================
% BEGIN INTRODUCTION
%==========================================================================

%% Starting page numbering here
\pagenumbering{arabic}

\chapter{Arquitetura do Serviço}
    \section{Diagrama de Arquitetura do Serviço}
        \begin{figure}[!ht]
            \centering
            \includegraphics[scale=0.7]{diagrams/architecture.png}
            \caption{\small Arquitetura do Serviço}
            \label{fig:1.1}
        \end{figure}
    \section{Descrição dos Módulos Desenvolvidos}
        \subsection{Orquestrador (\textit{orchestrator.c})}
            O \textit{orchestrator.c} é o módulo principal do serviço, sendo responsável por:
            \begin{itemize}
                \item Leitura e validação dos argumentos da linha de comandos;
                \item Criar o seu \textit{FIFO} para receber pedidos dos clientes;
                \item Lidar adequadamente com os pedidos recebidos dos clientes;
                \item Caso o número máximo de tarefas em paralelo seja atingido,
                    colocar os pedidos de execução em espera;
                \item Caso necessário, enviar as respostas dos pedidos dos clientes pelos seus respetivos \textit{FIFOs}.
            \end{itemize}

            O orquestrador encarrega-se de quatro tipos de pedidos:
            \begin{itemize}
                \item \textit{\textbf{EXECUTE}} - Pedido para executar um comando.
                    Se o número máximo de tarefas em paralelo for atingido,
                    o orquestrador coloca o pedido em espera,
                    de seguida envia uma resposta ao cliente com o número da tarefa
                    e o estado do pedido (“a executar” ou “agendada”).

                \item \textit{\textbf{COMPLETED}} - Assim que um comando terminar a sua execução,
                    o orquestrador recebe um pedido deste tipo enviado pelo processo-pai do processo-filho
                    que executou o comando, e executará, com base na política de escalonamento utilizada,
                    o próximo comando em espera, se algum existir.

                \item \textit{\textbf{STATUS}} - Pedido para consultar o estado das tarefas.
                    O orquestrador utiliza um processo para gerar e enviar a resposta ao cliente.
                    Deste modo, o servidor pode continuar a receber e lidar com pedidos de outros
                    clientes, sem interrupções.

                \item \textit{\textbf{KILL}} - Pedido para terminar o servidor.
                    Quando o orquestrador recebe este pedido,
                    termina a sua execução, porém, antes de terminar,
                    guarda o número de tarefa atual,
                    fecha os descritores de \textit{FIFOs} abertos,
                    e liberta a memória alocada para as tarefas em execução e em espera.
            \end{itemize}
            O orquestrador continua a correr até receber o pedido \textit{kill} de um cliente,
            ou até receber um sinal do sistema operativo para interromper/terminar.

            Para uma ilustração mais detalhada da comunicação entre o servidor e o cliente,
            por favor consulte os seguintes diagramas de comunicação nas figuras
            \textit{\ref{fig:1.2}} e \textit{\ref{fig:1.3}}, do subcapítulo \textit{\nameref{sec:1.3}}.

        \subsection{Cliente (\textit{client.c})}
            O módulo \textit{client.c} é responsável por, após a leitura e validação dos argumentos da linha de comandos,
            enviar pedidos - \textit{execute}, \textit{status} e \textit{kill} - para o \textit{FIFO} do
            servidor \textit{orchestrator}, e por receber as respostas do servidor num \textit{FIFO} criado para o efeito,
            no caso do pedido enviado ser do tipo \textit{execute} ou \textit{status}. O nome do \textit{FIFO} do cliente
            é gerado através do seu PID, e é enviado no próprio pedido para o servidor.
        \subsection{Pedido (\textit{request.c})}
            Neste módulo, é definida a estrutura de dados que representa um pedido, constituída por:
            \begin{itemize}
                \item \textbf{\textit{int type}} - Tipo do pedido
                    (\textit{EXECUTE}, \textit{COMPLETED}, \textit{STATUS} ou \textit{KILL});
                \item \textbf{\textit{int est\_time}} - Tempo estimado de execução da tarefa
                    ou a prioridade da tarefa, dependendo da política de escalonamento em uso;
                \item \textbf{\textit{char command[MAX\_CMD\_SIZE]}} - Comando a executar, caso seja um pedido de execução;
                \item \textbf{\textit{bool is\_piped}} - Indica se o comando é encadeado, caso seja um pedido de execução;
                \item \textbf{\textit{unsigned int task\_nr}} - Número da tarefa, atribuído pelo orquestrador, caso seja um pedido de execução;
                \item \textbf{\textit{char client\_fifo[CLIENT\_FIFO\_SIZE]}} - Nome do \textit{FIFO} do cliente,
                    para onde o orquestrador enviará a resposta, caso seja um pedido de execução ou de estado.
            \end{itemize}

            Este módulo garante o encapsulamento de dados, este é assegurado através da definição da
            \textit{struct request} dentro do ficheiro \textit{request.c}, e a sua manipulação é feita através
            de funções definidas neste módulo, como, por exemplo, a função \textit{create\_request}, e as funções
            de \textit{getters} e de \textit{setters}.

            Também oferece outras funções de utlidade, como a função de clonagem de pedidos, e impressão de pedidos,
            utlizadas no orquestrador.

        \subsection{Comando (\textit{command.c})}
            O módulo \textit{command.c} é chamado pelo orquestrador para executar os comandos enviados pelos clientes.

            Este módulo é responsável por:
            \begin{itemize}
                \item Executar comandos, \textit{single} ou \textit{piped}, utilizando \textit{process forking},
                    fazendo o respetivo \textit{parsing} dos comandos;
                \item Lida com o redirecionamento do \textit{stdout} e \textit{stderr} para os respetivos ficheiros;
                \item Encarrega-se da escrita do tempo de espera, bem como o de execução das tarefas para ficheiro;
                \item Escreve o número da tarefa, o comando executado e o tempo total para o ficheiro de histórico.
            \end{itemize}

            Primeiramente, todos os ficheiros necessários são criados - nomeadamente a diretoria de
            \textit{output} para os ficheiros relativos à tarefa, como o ficheiro de \textit{output},
            o de erros e o do tempo total.
            Também prepara a abertura do ficheiro de histórico, onde serão escritos os dados como o número da tarefa,
            o comando executado e o tempo total, após a execução do comando.

            De seguida, é criado um processo utilizado para criar outro processo-filho que executará o comando.
            O processo-pai aguardará que o seu processo-filho termine a execução do comando, e, após a sua execução,
            escreverá o número da tarefa, o comando executado e o tempo total, em milissegundos, para o ficheiro de histórico,
            e notificará o orquestrador que a tarefa foi concluída.

            São criados dois processos para que o processo-pai interior possa esperar pela execução do comando,
            sem bloquear a execução do orquestrador, permitindo a este continuar a lidar com outros pedidos de clientes.

            O redirecionamento do \textit{stdout} e \textit{stderr} para os respetivos ficheiros é feito
            através da utilização da função \textit{dup2}.

            No encadeamento de comandos, o redirecionamento é feito de forma a que o \textit{stdout} do comando
            anterior seja redirecionado para o \textit{stdin} do comando seguinte, e assim sucessivamente,
            até ao último comando, exclusivamente. Ao longo
            da cadeia de comandos, é feito o redirecionamento do \textit{stderr} para o ficheiro de erros, garantindo
            a similaridade com o comportamento nativo da \textit{shell}. No último comando, o redirecionamento do
            \textit{stdout} é feito para o ficheiro de \textit{output}.

        \subsection{Número de Tarefa (\textit{task\_nr.c})}
            O módulo \textit{task\_nr.c} proporciona funções para salvar e carregar o número
            de tarefa do respetivo ficheiro com a designação definida pela
            \textit{macro TASK\_NR\_FILENAME}, neste caso, "task\_number".

            Se, ao carregar o ficheiro do número da tarefa, este não existir,
            o número da tarefa é inicializado a 1,
            caso contrário, o número da tarefa é incrementado em 1.

            Este módulo é utilizado pelo orquestrador na sua inicialização para carregar o
            número da tarefa, e no seu término para o salvar.

    \clearpage
    \section{Diagramas de Comunicação entre Servidor e Cliente}
        \label{sec:1.3}
        Nos seguintes três subcapítulos, são apresentados diagramas que ilustram a comunicação
        entre o servidor e o cliente, consoante diferentes tipos de pedidos.
        \subsection{Execução de Tarefas (\textit{execute})}
                \begin{figure}[!ht]
                    \centering
                    \includegraphics[width=\textwidth]{diagrams/execute.png}
                    \caption{\small Diagrama de Comunicação para Execução de Tarefas}
                    \label{fig:1.2}
                \end{figure}
        \subsection{Estado de Tarefas (\textit{status})}
            \begin{figure}[!ht]
                \centering
                \includegraphics[width=\textwidth]{diagrams/status.png}
                \caption{\small Diagrama de Comunicação para Estado de Tarefas}
                \label{fig:1.3}
            \end{figure}
        \subsection{Terminação do Servidor (\textit{kill})}
            Na terminação do servidor, o cliente envia um pedido \textit{kill} para o \textit{FIFO} do servidor.
            De seguida o servidor guarda o número atual da tarefa de forma persistente,
            fecha os descritores de \textit{FIFOs} abertos,
            liberta a memória alocada para as tarefas em execução e em espera, e, por fim, termina a
            sua execução.

\begin{minipage}{\textwidth}
\chapter{Argumentos da Interface de Linha de Comandos}
    \section{\textit{Orchestrator}}
\begin{lstlisting}
Usage: orchestrator <output_dir> <parallel_tasks> [sched_policy]

Options:
    <output_dir>         Directory to store task output and history files
    <parallel_tasks>     Maximum number of tasks running in parallel
    [sched_policy]       Scheduling policy (FCFS, SJF, PES) default: SJF
\end{lstlisting}
    \section{\textit{Client}}
\begin{lstlisting}
Usage: client <option [args]>

Options:
    execute <est_time|priority> <-u|-p> "<command [args]>"            Execute a command
        est_time    Estimated time of execution, in case scheduling policy is SJF
        priority    Priority of the task, in case scheduling policy is PES
        -u          Unpiped command
        -p          Piped command
        command     Command to execute. Maximum size: 300 bytes
    status                                                            Status of the tasks
    kill                                                              Terminate the server

Note: The estimated time or priority is irrelevant if the scheduling policy is FCFS, but it must be provided as an argument.
\end{lstlisting}
\end{minipage}

\newgeometry{top=1cm,left=3cm,right=3cm,bottom=3cm}
\chapter{Avaliação de Políticas de Escalonamento}
    No trabalho desenvolvido, foram implementadas três políticas de escalonamento de tarefas.
    Estas serão aprofundadas nas secções seguintes, assim como as suas avaliações práticas.
    Por favor consulte o anexo \nameref{anexo:2} para a visualização da implementação destas.

    Para garantir a utilização correta das políticas de escalonamento,
    assume-se que, nos testes de avaliação, o tempo de execução das tarefas
    não difere - de forma a não afetar a ordem fornecida pelos algoritmos
    de escalonamento - dos tempos estimados fornecidos pelo cliente.
    Se estes tempos diferirem desta forma, a eficiência máxima das políticas de escalonamento
    não pode ser garantida, uma vez que a ordem de execução das tarefas não corresponderia
    à ordem mais eficiente, no caso das políticas \textit{SJF} e \textit{PES}.

    Nestas avaliações, assume-se que o número máximo de tarefas em paralelo é 2,
    que o tempo de resposta do servidor é 0 ms,
    e são considerados os seguintes quatro comandos, em fila de espera,
    com os respetivos tempos estimados e de execução (valores simbólicos):

    \begin{table}[!ht]
        \centering
        \begin{tabular}{|c|c|c|c|c|c|}
            \hline
            \rowcolor{gray!20!white}
            \textbf{Nr.} & \textbf{Comando} & \textbf{T. Estimado} & \textbf{T. Execução} \\
            \hline
            1 & \textit{ls -l | wc -l} & 100 &  90 \\
            \hline
            2 & \textit{echo "Hello World"} &  50 &  60 \\
            \hline
            3 & \textit{make bin/orchestrator} & 150 & 145 \\
            \hline
            4 & \textit{ps aux} &  80 &  85 \\
            \hline
        \end{tabular}
        \caption{\small Comandos a Executar para Avaliação das Políticas de Escalonamento}
    \end{table}

    Fórmulas utilizadas para a avaliação das políticas de escalonamento:

    \begin{minipage}{\textwidth}
        \centering
        $T_{\text{Total}} = T_{\text{Espera}} + T_{\text{Execução}}$
        \quad e \quad
        $T_{\text{Médio por tarefa}} = \frac{\sum_{i=1}^{n} T_{\text{Total}_{i}}}{n}$
        \quad e \quad
        $T_{\text{Total das tarefas}} = max_{i=1}^{n} T_{\text{Total}_{i}}$

        \vspace{0.2cm}
        onde $n$ é o número de tarefas e $T$ é o tempo em milissegundos.
    \end{minipage}

    \section{\textbf{FCFS} - \textit{First Come First Served}}
        Nesta política, as tarefas são executadas por ordem de chegada, o que não é
        eficiente em termos de tempo de espera e execução. No entanto, é uma política
        simples e fácil de implementar.

        \begin{table}[!ht]
            \centering
            \begin{tabular}{|c|c|c|c|c|c|}
                \hline
                \textbf{T. Total (Estimado)} & \diagbox{\textbf{T. Total (Execução)}}{\textbf{Nr.}} & \textbf{1} & \textbf{2} & \textbf{3} & \textbf{4} \\
                \hline
                        0 &        0 & \Exec & \Exec & \Wait & \Wait \\
                \hline
                       50 &       60 & \Exec & \End  & \Exec & \Wait \\
                \hline
                      100 &       90 & \End  & \End  & \Exec & \Exec \\
                \hline
                100 +  80 & 90 +  85 & \End  & \End  & \Exec & \End  \\
                \hline
                 50 + 150 & 60 + 145 & \End  & \End  & \End  & \End  \\
                \hline
            \end{tabular}
            \caption{\small Execução das Tarefas com a Política \textit{FCFS}}
        \end{table}

        \begin{minipage}{\textwidth}
            \centering
            $T_{\text{Médio por tarefa}} = \frac{60 + 90 + (90 + 85) + (60 + 145)}{4} = \frac{530}{4} = 132.5 \approx 133 \text{ ms}$

            \vspace{0.2cm}
            $T_{\text{Total das tarefas}} = max(60, 90, 90 + 85, 60 + 145) = 205 \text{ ms}$
        \end{minipage}

    \loadgeometry{default}
    \section{\textbf{SJF} - \textit{Shortest Job First}}
        A política \textit{Shortest Job First} é uma política de escalonamento não preemptiva
        que seleciona a tarefa com o menor tempo de execução. Este tempo é passado como
        argumento na opção \textit{execute} do cliente e repesenta uma estimativa do tempo
        que a tarefa demorará a ser executada.

        Esta política é eficiente em termos de tempo de espera e execução, mas pode levar
        a situações de \textit{starvation}, que ocorrem quando tarefas com tempos de execução
        maiores são sempre adiadas, e, num caso extremo, quando surgem constantemente tarefas
        com tempos de execução menores, estas de maior duração nunca serão executadas.

        Existem várias maneiras de lidar com situações de \textit{starvation}, como, por exemplo,
        a criação de uma especificação de tempo máximo de espera para cada tarefa, ou a
        definição de um número máximo de tarefas que podem ser executadas antes de uma
        tarefa com tempo de execução maior. Outra solução seria a implementação de uma
        política de escalonamento preemptiva, permitindo esta a interrupção de tarefas em
        execução para dar lugar a tarefas com tempos de execução menores.

        \begin{table}[!ht]
            \centering
            \begin{tabular}{|c|c|c|c|c|c|}
                \hline
                \textbf{T. Total (Estimado)} & \diagbox{\textbf{T. Total (Execução)}}{\textbf{Nr.}} & \textbf{1} & \textbf{2} & \textbf{3} & \textbf{4} \\
                \hline
                       0 &        0 & \Wait & \Exec & \Wait & \Exec \\
                \hline
                      50 &       60 & \Exec & \End  & \Wait & \Exec \\
                \hline
                      80 &       85 & \Exec & \End  & \Exec & \End  \\
                \hline
                50 + 100 & 60 +  90 & \End  & \End  & \Exec & \End  \\
                \hline
                80 + 150 & 85 + 145 & \End  & \End  & \End  & \End  \\
                \hline
            \end{tabular}
            \caption{\small Execução das Tarefas com a Política \textit{SJF}}
        \end{table}

        \begin{minipage}{\textwidth}
            \centering
            $T_{\text{Médio por tarefa}} = \frac{60 + 85 + (60 + 90) + (85 + 145)}{4} = \frac{525}{4} = 131.25 \approx 131 \text{ ms}$

            \vspace{0.2cm}
            $T_{\text{Total das tarefas}} = max(60, 85, 60 + 90, 85 + 145) = 230 \text{ ms}$
        \end{minipage}

        A política \textit{SJF} garante um tempo médio por tarefa menor em relação à política \textit{FCFS},
        com a exceção de quando a ordem de execução das tarefas não varia em relação à ordem de chegada,
        tendo estes tempos iguais.

        Podemos observar que, apesar da política \textit{SJF} ser mais demorada a executar
        as tarefas na totalidade em relação à política \textit{FCFS}, esta consegue ser mais eficiente
        em termos do tempo médio por tarefa, tendo assim um melhor aproveitamento do paralelismo.

    \section{\textbf{PES} - \textit{Priority Escalation Scheduling}}
        Por fim, foi implementada a política não preemptiva \textit{Priority Escalation Scheduling}.
        Nesta, as tarefas são executadas
        de acordo com a sua prioridade, que é definida pelo cliente na opção \textit{execute},
        simliarmente como o tempo estimado que é passado como argumento na política \textit{SJF}.

        As tarefas com maior prioridade são executadas primeiro.
        As prioridades ficam a critério do cliente,
        possibilitando uma maior versatilidade na execução de tarefas, permitindo ao cliente
        definir a prioridade adequada para cada tarefa que deseja executar. No entanto, esta
        flexibilidade não garante a eficiência da execução das tarefas, caso estas não sejam
        corretamente priorizadas.

        \clearpage
        Devido a esta política ser dependente da prioridade das tarefas, a sua eficiência
        é variável, podendo ser menos eficiente que a política \textit{SJF} em situações
        em que as prioridades das tarefas não são corretamente definidas. No entanto, esta
        política sendo determinística, em situações em que as prioridades são corretamente
        definidas, pode ser tão eficiente como a política \textit{SJF}.

        Para atingir uma maior eficiência, era necessário a implementação de uma política
        de escalonamento preemptiva, que permitisse a interrupção de tarefas em execução
        para dar lugar a tarefas com prioridades mais altas ou com tempos de execução
        menores.

\begin{minipage}{\textwidth}
\chapter{Testes Desenvolvidos}
    A testagem do serviço é fundamental para garantir o seu correto funcionamento.

    \vspace{1em}

    Para agilizar este processo, para além da testagem manual,
    foram desenvolvidos os seguintes testes automáticos:

    \begin{itemize}
        \item \textbf{\textit{task\_output\_and\_error.sh}} - Testa a escrita correta dos ficheiros
            de \textit{output}, de erros e do ficheiro de tempo total;
        \item \textbf{\textit{overload.sh}} - Testa o servidor de forma a superar a sua capacidade máxima
            de pedidos de execução;
        \item \textbf{\textit{policies.sh}} - Testa as três políticas de escalonamento implementadas;
        \item \textbf{\textit{cmds\_valid.sh}} - Testa a execução de comandos válidos, tanto únicos como
            em \textit{pipe}, em modo \textit{batch};
        \item \textbf{\textit{cmds\_invalid.sh}} - Testa a execução de comandos inválidos, tanto únicos
            como em \textit{pipe}, em modo \textit{batch};
    \end{itemize}

    \vspace{1em}

    Os testes \textbf{\textit{cmds\_valid.sh}} e \textbf{\textit{cmds\_invalid.sh}} utilizam os ficheiros
    \textbf{\textit{cmds\_valid.txt}} e \textbf{\textit{cmds\_invalid.txt}}, respetivamente, para a definição
    dos comandos a executar. Estes são importantes no processo de testagem,
    pois comparam o comportamento da execução de comandos pelo servidor
    com o comportamento nativo da \textit{shell}, e permitiram à equipa de trabalho
    garantir a correta execução destes comandos e a sua similaridade com o comportamento
    nativo da \textit{shell}.

    \vspace{1em}

    O teste \textbf{\textit{policies.sh}} e, também, o \textbf{\textit{overload.sh}} são testes que permitem
    avaliar o comportamento do servidor com diferentes configurações de paralelização,
    garantindo avaliar o funcionamento correto do paralelismo do servidor, assim como a implementação
    correta e eficiência das políticas de escalonamento.

    \vspace{1em}

    Assim, concluímos o funcionamento correto do serviço após a execução deste conjunto de testes,
    para diferentes configurações de paralelização do servidor, assim como para diferentes
    usagens do cliente.

    \vspace{1em}

    Os \textit{scripts shell} de testagem estão localizados na diretoria \textbf{tests/}
    na raíz do projeto.
\end{minipage}

%==========================================================================
% BEGIN CONCLUSÃO
%==========================================================================

\chapter{Conclusão}
    Após o desenvolvimento deste serviço, foram cumpridos todos
    os objetivos propostos no enunciado do trabalho, desde a implementação de
    uma interface de linha de comandos tanto para o servidor como para o cliente -
    que permite a execução de tarefas do utilizador de forma assíncrona com a
    possibilidade de encadeamento de programas com \textit{pipes} e a
    consulta do estado das tarefas -, assim como a implementação de um ficheiro
    \textit{log} para guardar as tarefas executadas, bem como a implementação de um
    sistema com várias políticas de escalonamento e a avaliação prática destas.
    Também foi desenvolvido um conjunto de testes que permitem verificar o correto
    funcionamento do serviço.

    Após a afirmação da possibilidade da terminação ou interrupção do servidor
    pelo professor regente, foi despertada a curiosidade de como o servidor
    poderia lidar com estas situações.

    Num estudo apronfundado desta unidade curricular, uma solução seria a
    implementação de um sistema de interrupções no servidor, o que permitiria
    a este lidar com situações de terminação ou interrupção de forma mais
    eficiente através da utlização dos sinais do sistema operativo para lidar com estas
    situações, evitando a perda de tarefas agendadas ou em execução.

    Para tal, seria necessário lidar com tarefas em execução, guardando
    o estado destas em memória, e permitindo a sua correta terminação
    ou interrupção, assim como para as tarefas em espera. Estas poderiam
    ser guardadas numa fila de espera de forma persistente,
    que seria consultada sempre que uma tarefa terminasse,
    permitindo a execução da próxima tarefa na fila, similarmente
    com o que foi implementado.

    Em suma, o desenvolvimento deste serviço permitiu a aplicação prática
    dos conhecimentos adquiridos ao longo do semestre, assim como a
    exploração de novas funcionalidades e conceitos, que proporcionaram
    a implementação de um serviço robusto e versátil - serviço este que
    propicia a execução de tarefas de forma assíncrona, com a possibilidade de
    encadeamento de programas e a consulta do estado das tarefas.

%==========================================================================
% END CONCLUSÃO
%==========================================================================

%==========================================================================
% BEGIN ANEXOS
%==========================================================================
%
%% Why \addchap, instead of \chapter?
%% \addchap has no numbering but appears in table of contents.
\addchap{Anexos}

    \addsec{
        \href{https://migueltc13.github.io/OS-doc/}{\small \textcolor{blue}{[I] Documentação do Código Desenvolvido}}
        \label{anexo:1}
    }

    \quad Para uma versão em formato \textit{pdf} da documentação do código desenvolvido, por favor consulte o
    ficheiro \href{anexos/refman.pdf}{\textcolor{blue}{\textbf{\textit{doc/latex/refman.pdf}}}}
    localizado na diretoria raiz do projeto.

    \addsec{
        \href{anexos/policies.pdf}{\small \textcolor{blue}{[II] Implementação das Políticas de Escalonamento}}
        \label{anexo:2}
    }

%==========================================================================
% END ANEXOS
%==========================================================================
\end{document}
